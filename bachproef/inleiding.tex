% !TeX spellcheck = <none>
%%=============================================================================
%% Inleiding
%%=============================================================================

\chapter{\IfLanguageName{dutch}{Inleiding}{Introduction}}%
\label{ch:inleiding}

  PIM-systemen zijn alomtegenwoordig in de huidige e-commerce wereld. PIMLayer is een Gents bedrijf dat PIM-systemen aanbiedt. Ze bieden een standaard oplossing aan die bedrijven zelf kunnen aanpassen naar hun eigen noden en wensen. Ze mikken vooral op kleine Vlaamse KMO's die nog geen PIM-systeem hebben.  PIMLayer werkt met een taxomomie systeem waardoor er zeer performante en precieze queries kunnen worden op gesteld door de gebruiker. 

\section{\IfLanguageName{dutch}{Probleemstelling}{Problem Statement}}%
\label{sec:probleemstelling}

Het taxonomie systeem dat PIMLayer gebruikt is een zeer grote troef ten opzichte van andere aanbieders van PIM-systeme, maar het zorgt ook voor een uitdaging: alle labels moeten door de gebruiker zelf worden toegevoegd. Dit zorgt voor een grote work-load bij de klant zelf. Zoals eerder al vermeld bestaan het clienteel van PIMLayer vooral uit kleinere Vlaamse KMO's, die vaak weinig tot geen tijd hebben om zich bezig te houden met die taak.  De oplossing voor dit probleem zou een slimme onboarding-tool kunnen zijn die, gebruikmakend van machine learning, slimme suggesties kan doen naar de klant toe en eventueel zelf zelfstandig labels kan toevoegen aan het systeem.

\section{\IfLanguageName{dutch}{Onderzoeksvraag}{Research question}}%
\label{sec:onderzoeksvraag}

Welke aanbieder van Machine Learning as a Service (MLaaS) past het best bij deze use-case? Amazon Machine Learning services, Microsoft Azure Machine Learning Studio of Google Cloud Platform?

\section{\IfLanguageName{dutch}{Onderzoeksdoelstelling}{Research objective}}%
\label{sec:onderzoeksdoelstelling}

Na de vergelijkende studie tussen de 3 aanbieders van MLaaS zal het duidelijk zijn of het toevoegen van MLaaS überhaupt een oplossing biedt voor de grote werklast die het onboarding proces (specifiek het toevoegen van de labels). Het zal ook duidelijk zijn welke oplossing het best presteerd voor deze specifieke use-case. Deze zal dan kunnen worden toegevoegd aan de slimme onboarding tool van PIMLayer.

\section{\IfLanguageName{dutch}{Opzet van deze bachelorproef}{Structure of this bachelor thesis}}%
\label{sec:opzet-bachelorproef}

% Het is gebruikelijk aan het einde van de inleiding een overzicht te
% geven van de opbouw van de rest van de tekst. Deze sectie bevat al een aanzet
% die je kan aanvullen/aanpassen in functie van je eigen tekst.

De rest van deze bachelorproef is als volgt opgebouwd:

In Hoofdstuk~\ref{ch:stand-van-zaken} wordt een overzicht gegeven van de stand van zaken binnen het onderzoeksdomein, op basis van een literatuurstudie.

In Hoofdstuk~\ref{ch:methodologie} wordt de methodologie toegelicht en worden de gebruikte onderzoekstechnieken besproken om een antwoord te kunnen formuleren op de onderzoeksvragen.

% TODO: Vul hier aan voor je eigen hoofstukken, één of twee zinnen per hoofdstuk

In Hoofdstuk~\ref{ch:conclusie}, tenslotte, wordt de conclusie gegeven en een antwoord geformuleerd op de onderzoeksvragen. Daarbij wordt ook een aanzet gegeven voor toekomstig onderzoek binnen dit domein.