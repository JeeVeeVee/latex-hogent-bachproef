%==============================================================================
% Sjabloon onderzoeksvoorstel bachproef
%==============================================================================
% Gebaseerd op document class `hogent-article'
% zie <https://github.com/HoGentTIN/latex-hogent-article>

% Voor een voorstel in het Engels: voeg de documentclass-optie [english] toe.
% Let op: kan enkel na toestemming van de bachelorproefcoördinator!
\documentclass{hogent-article}

% Invoegen bibliografiebestand
\addbibresource{voorstel.bib}

% Informatie over de opleiding, het vak en soort opdracht
\studyprogramme{Professionele bachelor toegepaste informatica}
\course{Bachelorproef}
\assignmenttype{Onderzoeksvoorstel}
% Voor een voorstel in het Engels, haal de volgende 3 regels uit commentaar
% \studyprogramme{Bachelor of applied information technology}
% \course{Bachelor thesis}
% \assignmenttype{Research proposal}

\academicyear{2022-2023} % TODO: pas het academiejaar aan

% TODO: Werktitel
\title{Toevoegen van Machine Learning aan taxonomie-systeem PIMLayer, een vergelijkende studie tussen Amazon Machine Learning services, Microsoft Azure Machine Learning Studio and Google Cloud Platform}

% TODO: Studentnaam en emailadres invullen
\author{Jules Vervaeke}
\email{jules.vervaeke@student.hogent.be}

% TODO: Medestudent
% Gaat het om een bachelorproef in samenwerking met een student in een andere
% opleiding? Geef dan de naam en emailadres hier
% \author{Yasmine Alaoui (naam opleiding)}
% \email{yasmine.alaoui@student.hogent.be}

% TODO: Geef de co-promotor op
\supervisor[Co-promotor]{Pascal Synaeve (Aware NV, \href{mailto:pascal.synaeve@aware.be}{pascal.synaeve@aware.bee})}

% Binnen welke specialisatierichting uit 3TI situeert dit onderzoek zich?
% Kies uit deze lijst:
%
% - Mobile \& Enterprise development
% - AI \& Data Engineering
% - Functional \& Business Analysis
% - System \& Network Administrator
% - Mainframe Expert
% - Als het onderzoek niet past binnen een van deze domeinen specifieer je deze
%   zelf
%
\specialisation{Mobile \& Enterprise development}
\keywords{PIM, Machine Learning, e-commerce, taxonomie}

\begin{document}

\begin{abstract}
  PIM-systemen zijn alomtegenwoordig in de huidige e-commerce wereld. PIMLayer is een Gents bedrijf dat PIM-systemen aanbiedt. Hun implementaties maken gebruik van een taxonomie systeem die de gebruikers toelaat om snelle en performante query's op te stellen en uit te voeren. Vooraleer ze gebruik kunnen maken van deze feature moeten alle entiteiten en assets die aanwezig zijn in het PIM-systeem wel voorzien worden van de juiste tags, dit is een proces dat de gebruikers zelf moeten doen en dat vaak een  grote werklast met zich meebrengt. Om dit probleem op te lossen zou er gebruikt kunnen gemaakt worden van een slimme onboarding tool die, gebruikmakend van machine learning, slimme suggesties kan geven aan de gebruiker en eventueel ook zelf tags kan toevoegen. Voor de implementatie van deze tool zal er gebruik woorden gemaakt van een MLaaS ( Machine Learning as a Service). Het doel van deze bacherlorproef is een vergelijkende studie maken tussen 3 aanbieders van MLaaS voor de specifieke use case van het taxonomie systeem van PIMLayer. Deze vergelijking zal worden gemaakt door voor elke optie een oplossing op te stellen en deze vervolgens te testen. De oplossingen zullen worden getest op hun leercurve, snelheid en accuraatheid ten opzichte van een bestaande dataset van PIMLayer. Op basis van deze vergelijking kan er dan een keuze worden gemaakt voor de optie die effectief gebruikt kan worden in de implementaties van PIMLayer. Onderzoek uit het verleden laat vermoeden dat Amazon de meest interessante oplossing aanbiedt voor de use-case.
\end{abstract}

\tableofcontents

% De hoofdtekst van het voorstel zit in een apart bestand, zodat het makkelijk
% kan opgenomen worden in de bijlagen van de bachelorproef zelf.
%---------- Inleiding ---------------------------------------------------------

\section{Introductie}%
\label{sec:introductie}

PIM (Product Information Management) systemen zijn alomtegenwoordig in de e-commerce wereld. Ze focussen op het onderhouden, verrijken en beheren van alle info die te maken heeft met de producten van een bedrijf. Het doen van dit proces is het creëren van een single source of truth, die dan kan gebruikt worden voor sales en marketing doeleinden. \autocite{Matilla2018}

PIMLayer een Gents bedrijf dat PIM-systemen aanbiedt. Hun implementaties werken met een taxonomie systeem dat toelaat om elk product en item op verschillende en meerdere niveaus van tags te voorzien. \autocite{AwareNV2022} Dit systeem zorgt ervoor dat de gebruiker heel specifieke en performante query's kan opstellen binnen het systeem. Een gebrek van deze feature is dat alle tags door de gebruiker zelf moeten worden toegevoegd aan alle producten en items in het systeem.


De oplossing voor dit probleem zou een slimme onboarding tool zijn die, gebruik makend van Machine Learning, slimme suggesties kan doen voor de tags die bij een bepaald item of product horen op basis van de data die al aanwezig is in het PIM systeem, en op termijn misschien zelfs zelfstandig tags kan toevoegen. De implementatie van de tool zou gebruik maken van een MLaaS (Machine Learning as a Service)  Het doel van deze bachelorproef is om 3 verschillende aanbieders van MLaaS: Amazon Machine Learning services, Microsoft Azure Machine Learning Studio en Google Cloud Platform, te vergelijken op basis van hoe snel ze een bepaald succespercentage kunnen behalen. 

Op basis van de resultaten van deze bachelorproef kan de best passende MLaaS worden geïntegreerd  in PIMLayer.



%---------- Stand van zaken ---------------------------------------------------

\section{State-of-the-art}%
\label{sec:state-of-the-art}

\subsection{Machine Learning}
\textcite{Hu2013} beschrijft het doel van machine learning als het ontwerpen en ontwikkelen van algoritmes die systemen toelaten om op basis van data, ervaring en training zichzelf aan te passen. 
\subsubsection{Algoritmes}
Machine learning algoritmes kunnen worden verdeeld op basis van hoe hun proces er uitziet :
\begin{itemize}
	\item \textbf{Supervised Learning}: het algoritme krijgt een gelabelde dataset om te trainen, op basis daarvan wordt een model opgebouwd dat niet gelabelde datasets kan classificeren. Achteraf moet een 'supervisor' hoeveel procent van de toegekende labels correct waren. 
	\item \textbf{Unsupervised learning}: het algoritme vertrekt vanuit een niet gelabelde dataset en gaat zelf op zoek naar patronen in de dataset. Op basis van de patronen maakt het algoritme zelf een classificatie.
	\item \textbf{Semi-supervised learning}: een hybride van supervised en unsupervised learning.
	\item \textbf{Reinforcement learing}: het algoritme leert hoe het zich moet gedragen op basis van een gegeven feit. Elke actie heeft een invloed op het model, dat op zijn beurt kan terugkoppelen naar het model om zo het gedrag te beïnvloeden.
	\item \textbf{Transduction}: gelijkaardig aan supervised learing, maar genereert niet per se een functie, het probeert de nieuwe outputs te voorspellen op basis van training inputs, training outputs en nieuwe inputs.
	\item \textbf{Learning to learn}: het algoritme leert zichzelf gedrag aan gebaseerd op vorige ervaring.
\end{itemize}
\autocite{Zhang2010}


\subsubsection{Machine Learning as a Service (MLaaS)}
MLaaS is de verzameltern voor de verzameling van cloud-based platforms die machine learing gebruiken om oplossingen aan te bieden. \autocite{Onose2022}. Deze oplossing kunnen meerdere vormen aannemen: 
\begin{itemize}
	\item voorspellende analyse voor verschillende use cases
	\iten trainen en afstellen van een model
	\item implemenatie van een model
\end{itemize}

MLaaS heeft ook een aantal gebreken, zo heb je zelf niet altijd controle over de gebruikte algoritmes en parameters en het is niet geschikt voor gevoelige data.

\subsubsection{eerder onderzoek}
\textcite{Madhuri2016} stelt dat de verschillen tussen Amazon Web Services (AWS) en Microsoft Azure klein zijn en dat de keuze voor de ene of de andere optie neerkomt op use-case en eigen voorkeur. \textcite{Pinto2018} gebruikte de volgende criteria om MLaaSes te vergelijken :
\begin{itemize}
	\item schaalbaarheid
	\item snelheid
	\item in welke mate dient de oplossing zijn oorspronkelijke doel
	\item bruikbaarheid
\end{itemize}
\textcite{Pallavi2020} vergeleek de AWS, Google en Azure oplossingen op algemeen vlak, en dus niet voor een specifieke usecase, maar concludeerde net als \textcite{Madhuri2016} dat de keuze afhankelijk is van de use-case en eigen voorkeur.

Uit \textcite{Pallavi2020} en \textcite{Madhuri2016} kunnen we volgende tabel besluiten: 
\begin{itemize}
	\item schaalbaarheid en het ondersteunen van meerdere SQL-varianten zijn de grootste troeven van AWS
	\item het feit dat Azure gemakkelijk in te passen is in een reeds bestaande Microsoft omgeving is een grote troef van Azure
	\item Google maakt gebruik van de meest performante systemen om zijn berekeningen te doen en heeft veel ingebouwde libraries
	\item Azure ondersteunt alleen AzureSQL
\end{itemize}


\subsection{PIM}
Het belang van Product Information Management (PIM) stijgt door het alsmaar verhoogde niveau van de technische complexiteit  van producten, die complexiteit zorgt ervoor namelijk voor dat er een overvloed aan informatie ontstaat die bovendien moeilijk up-to-date kan worden gehouden. \autocite{Fr_mling_2006} Vaak combineert een PIM-systeem data uit verschillende andere systemen en databronnen (zoals ERP's, databanken, excels, etc.) en bundelt het die tot een single-source of truth die bovendien altijd up-to-date is. De data van het PIM-systeem kan dan worden geconsumeerd door brand portals, product sheets, een webshop, ...



%---------- Methodologie ------------------------------------------------------
\section{Methodologie}%
\label{sec:methodologie}

Er wordt vertrokken vanuit een dataset van ongeveer 40 000 records, deze omvatten zowel entiteiten als assets (text-based) uit een reeds bestaand PIM-systeem. Hiervan beschouwen we de eerste 90\% als data die gebruikt mag worden om te trainen en de laatste 10\% als controle data. De controle data zal gebruikt worden om de nauwkeurigheid van de oplossing te testen. Om ervoor te zorgen dat we niet eindigen met de 4000 recentste records, waar nieuwe trends of fenomenen zich kunnen voordoen, halen we elk 10de record uit de dataset en voegen we deze toe aan de controle data.
Om de 3 opties te vergelijken, zullen de volgende stappen voor alle opties worden doorlopen:

\begin{enumerate}
	\item opzetten van een account
	\item configuratie van infrastructuur
	\item opladen van 10\% van de dataset
	\item laten ontwikkelen van oplossing
	\item testen van huidige oplossing tegen de controle data
	\item stappen 3) en 4) hernemen voor 20\%, 30\%, 40\%, 50\%, 60\%, 70\%, 80\% en 90\%
\end{enumerate}

Door verschillende percentages van de dataset te gebruiken krijgen we een beter zicht op de leercurve van een oplossing.

Hierna kunnen we de 3 opties vergelijken op: 
\begin{itemize}
	\item aantal records die de tool nodig heeft om een nauwkeurigheid van meer dan 50\%, 75\% en 90\% te bereiken
	\item tijd die de tool nodig heeft om een oplossing op te stellen
	\item evolutie verhouding tussen aantal records die te tool nodig had om te trainen en nauwkeurigheidspercentage
	\item tijd die tool nodig heeft om de controle data te verwerken (en eventueel of die verandert naarmate er meer data naar de tool wordt opgeladen).
	\item gebruiksvriendelijkheid van de tool en eventuele extra features, dit kan door het meten van de tijd die nodig is om een account aan te maken of om de omgeving te configuren.  
	\item prijs
\end{itemize}

Op basis van die vergelijking kan dan een keuze worden gemaakt voor de beste tool.

%---------- Verwachte resultaten ----------------------------------------------
\section{Verwacht resultaat, conclusie}%
\label{sec:verwachte_resultaten}

Er wordt verwacht dat de 3 tools niet significant zullen verschillen in performantie, de doorslaggevende factor(en) zullen dan waarschijnlijk prijs en gebruiksvriendelijkheid van de tool zijn. Het zou ook kunnen dat eventuele marketing doeleinden ook een rol spelen in de eindbeslissing. Amazon Machine Learning services bieden de meeste en meest geavanceerde extra-functionaliteiten aan en zijn ook zeer attractief op vlak van prijs. 

\printbibliography[heading=bibintoc]

\end{document}